\documentclass{article}[11pt]

% usepackages
\usepackage[ a4paper
           , textwidth  = 16.0cm
           , textheight = 25.0cm
           , headsep    =  0.25cm
           , voffset    =  0.3cm
           , footskip   =  1.25cm
           ]{geometry}

\usepackage{amsmath}
\usepackage{amssymb}
\usepackage{nicefrac}

% No intendation
\setlength\parindent{0pt}

\usepackage{hyperref}

\usepackage{siunitx}
\sisetup{
  per-mode=fraction,
  fraction-function=\tfrac
}

\usepackage{listings}
  \lstset{
    basicstyle=\ttfamily,
    escapeinside=||,
    xleftmargin=1cm
  }

\usepackage{float}

\usepackage{tikz}
  \usetikzlibrary{patterns}
  \usetikzlibrary{arrows.meta}
  \usetikzlibrary{shapes.misc}
  \usetikzlibrary{calc}

\usepackage{pgfplots}

\usepackage{cleveref}
\crefmultiformat{equation}{(#2#1#3)}{ and~(#2#1#3)}{, (#2#1#3)}{ and~(#2#1#3)}


\usepackage{acronym}
\usepackage[acronym,nonumberlist]{glossaries}
\makeglossaries
\newacronym{spice}{SPICE}{Simulation Program with Integrated Circuit Emphasis}
\newacronym{lef}{LEF}{Library Exchange Format}
\newacronym{dft}{DFT}{Discrete Fourier Transform}
\newacronym{dtft}{DTFT}{Discrete-Time Fourier Transform}
\newacronym{fft}{FFT}{Fast Fourier Transform}
\newacronym{mosfet}{MOSFET}{Metal–Oxide–Semiconductor Field-Effect Transistor}
\newacronym{clm}{CLM}{Channel Length Modulation}

% Section and subsection enumeration
\renewcommand{\thesection}{\Roman{section}.} 
\renewcommand{\thesubsection}{\thesection\Alph{subsection}}

% literature
\usepackage[backend=biber, isbn=true, sorting=none]{biblatex}
\addbibresource{./../../literature.bib}

% definitions
\def \whatis       {Notes}
\def \title        {Fuubar}

\def \author       {Matthias Schweikardt}

\def \authorMail   {mschweikardt@posteo.de}

\def \authorGithub {mschweikardt}

\def \license      {CC BY-SA 4.0}
\def \licenseUrl   {https://creativecommons.org/licenses/by-sa/4.0/}

\def \date         {nodate}

\def \pdfurl       {pdfurl}
\def \srcurl       {srcurl}


% Customize footer and header of document
\usepackage{fancyhdr}

% Access last page number
\usepackage{lastpage}

% Access last page number
\usepackage[thinc]{esdiff}

% Physics
\usepackage{physics}

% Comment environment
\usepackage{comment}

% Subcaptions
\usepackage{subcaption}

% Thicker lines in tables
\usepackage{booktabs}


% Defince title of document
\newcommand{\notetitle}{
  \begingroup
  \hypersetup{hidelinks}
  \thispagestyle{notefirst}
  \begin{center}
  \rule{\textwidth}{1pt}\\
  \medskip
  {\it \whatis}\\
  \bigskip
  {\LARGE \textbf{\title}}\\
  \medskip
  {\small \author}\\
  \rule{\textwidth}{0.5pt}\\
  {\small
    \begin{minipage}[t]{0.5\textwidth}
      \begin{tabular}[t]{ p{2.25cm} p{5.75cm}}
        Mail: & \href{mailto:\authorMail}{\tt{\authorMail}} \\
        Github: & \href{https://github.com/\authorGithub}{\tt{\authorGithub}} \\
      \end{tabular}
    \end{minipage}%
    %
    \begin{minipage}[t]{0.5\textwidth}
      \begin{tabular}[t]{ p{2.25cm} p{5.75cm} }
        Date: & \date  \\
        License: & \href{\licenseUrl}{\license}
      \end{tabular}
    \end{minipage}
  }%
  {\small
    \begin{minipage}[t]{\textwidth}
      \begin{tabular}[t]{ p{2.25cm} p{12cm}}
        Latest PDF: & \href{\pdfurl}{\tt{\pdfurl}} \\
        Latest Source: & \href{\srcurl}{\tt{\srcurl}}
      \end{tabular}
    \end{minipage}%
  }
  \bigskip
  \rule{\textwidth}{1pt}
  \end{center}
  \endgroup
}

% Header and footer on first page
\fancypagestyle{notefirst}{
  \fancyhf{}
  \renewcommand{\headrulewidth}{0pt}
  \renewcommand{\footrulewidth}{0pt}

  \fancyfoot[C]{\thepage/\pageref*{LastPage}}
}

% Header and footer on 2nd-last page
\fancypagestyle{noterest}{
  \fancyhf{}
  \renewcommand{\headrulewidth}{0.5pt}
  \renewcommand{\footrulewidth}{0.0pt}

  \fancyhead[L]{\author}
  \fancyhead[C]{\title}
  \fancyhead[R]{\date}

  \fancyfoot[C]{\thepage/\pageref*{LastPage}}
}
\pagestyle{noterest}


% Indentation in footnote
\makeatletter
\renewcommand\@makefntext[1]{\leftskip=2em\hskip-0.5em\@makefnmark#1}
\makeatother

% usepackages
\usepackage[ a4paper
           , textwidth  = 16.0cm
           , textheight = 25.0cm
           , headsep    =  0.25cm
           , voffset    =  0.3cm
           , footskip   =  1.25cm
           ]{geometry}

\usepackage{amsmath}
\usepackage{amssymb}
\usepackage{nicefrac}

% No intendation
\setlength\parindent{0pt}

\usepackage{hyperref}

\usepackage{siunitx}
\sisetup{
  per-mode=fraction,
  fraction-function=\tfrac
}

\usepackage{listings}
  \lstset{
    basicstyle=\ttfamily,
    escapeinside=||,
    xleftmargin=1cm
  }

\usepackage{float}

\usepackage{tikz}
  \usetikzlibrary{patterns}
  \usetikzlibrary{arrows.meta}
  \usetikzlibrary{shapes.misc}
  \usetikzlibrary{calc}

\usepackage{pgfplots}

\usepackage{cleveref}
\crefmultiformat{equation}{(#2#1#3)}{ and~(#2#1#3)}{, (#2#1#3)}{ and~(#2#1#3)}


\usepackage{acronym}
\usepackage[acronym,nonumberlist]{glossaries}
\makeglossaries
\newacronym{spice}{SPICE}{Simulation Program with Integrated Circuit Emphasis}
\newacronym{lef}{LEF}{Library Exchange Format}
\newacronym{dft}{DFT}{Discrete Fourier Transform}
\newacronym{dtft}{DTFT}{Discrete-Time Fourier Transform}
\newacronym{fft}{FFT}{Fast Fourier Transform}
\newacronym{mosfet}{MOSFET}{Metal–Oxide–Semiconductor Field-Effect Transistor}
\newacronym{clm}{CLM}{Channel Length Modulation}

% Section and subsection enumeration
\renewcommand{\thesection}{\Roman{section}.} 
\renewcommand{\thesubsection}{\thesection\Alph{subsection}}

% literature
\usepackage[backend=biber, isbn=true, sorting=none]{biblatex}
\addbibresource{./../../literature.bib}

% definitions
\def \whatis       {Notes}
\def \title        {Fuubar}

\def \author       {Matthias Schweikardt}

\def \authorMail   {mschweikardt@posteo.de}

\def \authorGithub {mschweikardt}

\def \license      {CC BY-SA 4.0}
\def \licenseUrl   {https://creativecommons.org/licenses/by-sa/4.0/}

\def \date         {nodate}

\def \pdfurl       {pdfurl}
\def \srcurl       {srcurl}


% Customize footer and header of document
\usepackage{fancyhdr}

% Access last page number
\usepackage{lastpage}

% Access last page number
\usepackage[thinc]{esdiff}

% Physics
\usepackage{physics}

% Comment environment
\usepackage{comment}

% Subcaptions
\usepackage{subcaption}

% Thicker lines in tables
\usepackage{booktabs}


% Defince title of document
\newcommand{\notetitle}{
  \begingroup
  \hypersetup{hidelinks}
  \thispagestyle{notefirst}
  \begin{center}
  \rule{\textwidth}{1pt}\\
  \medskip
  {\it \whatis}\\
  \bigskip
  {\LARGE \textbf{\title}}\\
  \medskip
  {\small \author}\\
  \rule{\textwidth}{0.5pt}\\
  {\small
    \begin{minipage}[t]{0.5\textwidth}
      \begin{tabular}[t]{ p{2.25cm} p{5.75cm}}
        Mail: & \href{mailto:\authorMail}{\tt{\authorMail}} \\
        Github: & \href{https://github.com/\authorGithub}{\tt{\authorGithub}} \\
      \end{tabular}
    \end{minipage}%
    %
    \begin{minipage}[t]{0.5\textwidth}
      \begin{tabular}[t]{ p{2.25cm} p{5.75cm} }
        Date: & \date  \\
        License: & \href{\licenseUrl}{\license}
      \end{tabular}
    \end{minipage}
  }%
  {\small
    \begin{minipage}[t]{\textwidth}
      \begin{tabular}[t]{ p{2.25cm} p{12cm}}
        Latest PDF: & \href{\pdfurl}{\tt{\pdfurl}} \\
        Latest Source: & \href{\srcurl}{\tt{\srcurl}}
      \end{tabular}
    \end{minipage}%
  }
  \bigskip
  \rule{\textwidth}{1pt}
  \end{center}
  \endgroup
}

% Header and footer on first page
\fancypagestyle{notefirst}{
  \fancyhf{}
  \renewcommand{\headrulewidth}{0pt}
  \renewcommand{\footrulewidth}{0pt}

  \fancyfoot[C]{\thepage/\pageref*{LastPage}}
}

% Header and footer on 2nd-last page
\fancypagestyle{noterest}{
  \fancyhf{}
  \renewcommand{\headrulewidth}{0.5pt}
  \renewcommand{\footrulewidth}{0.0pt}

  \fancyhead[L]{\author}
  \fancyhead[C]{\title}
  \fancyhead[R]{\date}

  \fancyfoot[C]{\thepage/\pageref*{LastPage}}
}
\pagestyle{noterest}


% Indentation in footnote
\makeatletter
\renewcommand\@makefntext[1]{\leftskip=2em\hskip-0.5em\@makefnmark#1}
\makeatother

\include{./../../tikzlib/figs/nmos-4t-testbench-a.tex}

\def \title  {BSIM-4}
\def \date   {May 18, 2025}

\def \pdfurl {https://mschweikardt.github.io/ee-notes/bsim-4.pdf}
\def \srcurl {https://github.com/mschweikardt/ee-notes/tree/main/notes/bsim-4}

\usepackage[scale=5]{draftwatermark}

\begin{document}

\notetitle

BSIM-4 has hundreds of model parameters, this why they are not explained in 
detail.
This model does in principle the same than the square-law 
model\cite{mosfet-square-law}, it maps the applied voltages on the drain 
current (I-V characteristic). 
These (complicated) equations that map the voltages on the current are made 
available by \gls{spice} simulators.
All major commercial and open-source \gls{spice} simulators provide a 
BSIM-4 model. 
However, they may be not exactly identical to that put on the official 
BSIM-4 web site \cite{liu-mosfetmodel-01} ,e.g., some RF simulators add an 
additional gate resistance to the model which is not present in the origin model.


\begin{figure}[H]
  \centering
  \begin{circuitikz}
    \NmosFourtTestbenchA
  \end{circuitikz}
  \caption{Definition of $V_{\mathrm{GS}}$, $V_{\mathrm{DS}}$, $V_{\mathrm{SB}}$ and $I_{\mathrm{D}}$}
  \label{fig:testbench}
\end{figure}


Several interim results (operating points) of these equations can be 
investigated by designers.
They allow a deeper insight in the behavior of the device and will help 
designers in size their circuits more easily.
An excerpt of a some operating points is shown in Tab. \ref{tab:oppoints}.

\begin{table}[H]
\centering
\caption{Some BSIM-4 operating points in Ngspice}
\begin{tabular}{ccl}
\toprule
\textbf{Identifier}  & \textbf{Variable}      & \textbf{Description}                                   \\ \midrule
\texttt{vth}         & $V_{\mathrm{TH}}$      & Threshold voltage                                      \\ 
\texttt{gm}          & $g_{\mathrm{m}}$       & Intrinsic transconductance                             \\ 
\texttt{gmbs}        & $g_{\mathrm{mb}}$      & Intrinsic backgate transconductance                    \\ 
\texttt{gds}         & $g_{\mathrm{DS}}$      & Intrinsic Drain-Source conductance                     \\ 
\texttt{vdsat}       & $V_{\mathrm{DS,sat}}$  & Saturation voltage                                     \\ 
\texttt{vds}         & $V_{\mathrm{DS,i}}$    & Intrinsic Drain-Source voltage                         \\ 
\texttt{vgs}         & $V_{\mathrm{GS,i}}$    & Intrinsic Gate-Source voltage                          \\ 
\texttt{qg}          & $Q_{\mathrm{G}}$       & Intrinsic Gate charge                                  \\ 
\texttt{qd}          & $Q_{\mathrm{D}}$       & Intrinsic Drain charge                                 \\ 
\texttt{qs}          & $Q_{\mathrm{S}}$       & Intrinsic Source charge                                \\ 
\texttt{qb}          & $Q_{\mathrm{B}}$       & Intrinsic Bulk charge                                  \\ 
\texttt{cgg}         & $C_{\mathrm{GG,i}}$    & Intrinsic Gate capacitance                             \\ 
\texttt{cdd}         & $C_{\mathrm{DD,i}}$    & Intrinsic Drain capacitance                            \\ 
\texttt{css}         & $C_{\mathrm{SS,i}}$    & Intrinsic Source capacitance                           \\ 
\texttt{cbb}         & $C_{\mathrm{BB,i}}$    & Intrinsic Bulk capacitance                             \\ 
\texttt{cds}         & $c_{\mathrm{DS,i}}$    & Intrinsic Drain-Source transcapacitance                \\ 
\texttt{csd}         & $c_{\mathrm{SD,i}}$    & Intrinsic Source-Drain transcapacitance                \\ 
\texttt{cdg}         & $c_{\mathrm{DG,i}}$    & Intrinsic Drain-Gate transcapacitance                  \\ 
\texttt{cgd}         & $c_{\mathrm{GD,i}}$    & Intrinsic Gate-Drain transcapacitance                  \\ 
\texttt{cdb}         & $c_{\mathrm{DB,i}}$    & Intrinsic Drain-Bulk transcapacitance                  \\ 
\texttt{cbd}         & $c_{\mathrm{BD,i}}$    & Intrinsic Bulk-Drain transcapacitance                  \\ 
\texttt{csg}         & $c_{\mathrm{SG,i}}$    & Intrinsic Source-Gate transcapacitance                 \\ 
\texttt{cgs}         & $c_{\mathrm{GS,i}}$    & Intrinsic Gate-Source transcapacitance                 \\
\texttt{cbg}         & $c_{\mathrm{BG,i}}$    & Intrinsic Source-Gate transcapacitance                 \\
\texttt{cgb}         & $c_{\mathrm{GB,i}}$    & Intrinsic Gate-Source transcapacitance                 \\
\texttt{csb}         & $c_{\mathrm{SB,i}}$    & Intrinsic Source-Bulk transcapacitance                 \\
\texttt{cbs}         & $c_{\mathrm{BS,i}}$    & Intrinsic Bulk-Source transcapacitance                 \\
\texttt{capbd}       & $C_{\mathrm{DB,j}}$    & Extrinsic Bulk-Drain junction capacitance              \\
\texttt{capbs}       & $C_{\mathrm{SB,j}}$    & Extrinsic Bulk-Source junction capacitance             \\
\texttt{cgso}        & $C_{\mathrm{GSO}}$     & Extrinsic Gate-Source overlap and fringing capacitance \\
\texttt{cgdo}        & $C_{\mathrm{GDO}}$     & Extrinsic Gate-Drain overlap and fringing capacitance  \\
\texttt{cgbo}        & $C_{\mathrm{GBO}}$     & Extrinsic Gate-Bulk overlap and fringing capacitance   \\ \toprule
\end{tabular}
\label{tab:oppoints}
\end{table}

Different \gls{spice} simulators have different naming conventions for 
operating points.
The nomenclature presented here correspond to the Ngspice circuit simulator.

\section{Intrinsic and Extrinsic}

The model of a MOSFET can be divided into an intrinsic and extrinsic part.

The intrinsic part describes the transistor itself (MOS structure), 
whereas the the extrinsic part models contact resistances, junctions and 
overlap capacitances.

Fig. \ref{fig:intr-extr} shows a model of a transistor 
(terminals $D$, $G$, $S$ and $B$) with an intrinsic MOS transistor 
(terminals $D'$, $G$, $S'$ and $B$) in the middle and various extrinsic 
components around.

\begin{figure}[H]
  \centering
  \begin{circuitikz}
    \ctikzset{bipoles/diode/width=0.3}
    \ctikzset{bipoles/diode/height=0.3}

    \node[nfet](mos) at (0,0) {};
    \node[]() at (0.2,0.2) {$M_{\mathrm{i}}$};

    \draw (mos.drain) to [short,- ] ++ (0,0.5) coordinate (draindiode)
                      to [short,-o] ++ (0,1)   node[right] (d2) {$D'$}
                      to [short,- ] ++ (0,1)   coordinate (c)
                      to [R,l_=$R_{\mathrm{D}}$,-o] ++ (0,2) node[right] (d) {$D$};

    \draw (mos.source) to [short,- ]  ++ (0,-0.5) coordinate (sourcediode)
                       to [short,-o]  ++ (0,-1)   node[right] (s2) {$S'$}
                       to [R,l=$R_{\mathrm{S}}$,-o] ++ (0,-2) node[right] (s) {$S$};

    \draw (mos.bulk) to [short,-] ++ (2,0) coordinate (b)
                     to [short,-] (draindiode-|b)
                     to [D,-*,l_=$D_{\mathrm{DB,j}}$] (draindiode);

    \draw (b) to [short,*-] (sourcediode-|b)
              to [D,-*,l=$D_{\mathrm{SB,j}}$] (sourcediode);

    \draw (mos.gate) to [short,-*] ++ (-1,0) coordinate (g)
                     to [short,-] (g|-draindiode) coordinate (x)
                     to [C,l=$C_{\mathrm{GDO}}$] (draindiode);

    \draw (g) to [short,-] (g|-sourcediode)
              to [C,l_=$C_{\mathrm{GSO}}$] (sourcediode); 

    \draw (x) to [short,*-] (x|-c) coordinate (x)
              to [C,l=$C_{\mathrm{GBO}}$] (x-|c)
              to [short,-] (x-|b)
              to [short,-*] (b|-draindiode);

    \draw (g) to [short,-o] ++ (-0.75,0) node[left] {$G$};
    \draw (b) to [short,-o] ++ ( 0.75,0) node[right] {$B$};
  \end{circuitikz}
  \caption{Intrinsic and Extrinsic components of a MOS transistor}
  \label{fig:intr-extr}
\end{figure}

$R_{\mathrm{D}}$ and $R_{\mathrm{S}}$ model the resistance of the Drain and 
Source regions. 
The diodes $D_{\mathrm{DB,j}}$ and $D_{\mathrm{SB,j}}$ belong to the junction 
of the Drain and Source regions.
They are reverse biased in normal operation mode of the transistor. 
The small signal model of these diodes consist of a capacitance and 
a conductance.
The overlap and fringing capacitance of the Gate with all other terminals are 
described with $C_{\mathrm{GSO}}$, $C_{\mathrm{GDO}}$ and $C_{\mathrm{GBO}}$.

\cite{bsim480-man}
\cite{ngspice-44}
\cite{tsividis-mosfetmodel-11}

\printbibliography

\end{document}