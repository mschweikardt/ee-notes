\documentclass{article}[11pt]

\usepackage{amsmath}
\usepackage{amssymb}
\usepackage{nicefrac}

\usepackage{pdflscape}

\usepackage{upgreek}

\usepackage{bashful}

% No intendation
\setlength\parindent{0pt}

\usepackage{hyperref}

\usepackage{siunitx}
\sisetup{
  per-mode=fraction,
  fraction-function=\tfrac
}

\usepackage{listings}
  \lstset{
    basicstyle=\ttfamily,
    escapeinside=||,
    xleftmargin=1cm
  }

\usepackage{float}

\usepackage{longtable}

\usepackage{multirow}

\usepackage{tikz}
  \usetikzlibrary{patterns}
  \usetikzlibrary{arrows.meta}
  \usetikzlibrary{shapes.misc}
  \usetikzlibrary{calc}

\usepackage{pgfplots}

\usepackage{cleveref}
\crefmultiformat{equation}{(#2#1#3)}{ and~(#2#1#3)}{, (#2#1#3)}{ and~(#2#1#3)}


\usepackage{acronym}
\usepackage[acronym,nonumberlist]{glossaries}
\glsdisablehyper
\makeglossaries
\newacronym{spice}{SPICE}{Simulation Program with Integrated Circuit Emphasis}
\newacronym{lef}{LEF}{Library Exchange Format}
\newacronym{dft}{DFT}{Discrete Fourier Transform}
\newacronym{dtft}{DTFT}{Discrete-Time Fourier Transform}
\newacronym{fft}{FFT}{Fast Fourier Transform}
\newacronym{mosfet}{MOSFET}{Metal–Oxide–Semiconductor Field-Effect Transistor}
\newacronym{clm}{CLM}{Channel Length Modulation}
\newacronym{de}{DE}{differential equation}
\newacronym{soi}{SOI}{silicon-on-insulator}
\newacronym{ldo}{LDO}{low-dropout regulator}
\newacronym{ota}{OTA}{operational-transconductance amplifier}
\newacronym{ofa}{OFA}{operational-floating amplifier}

% literature
\usepackage[ backend=biber
           , isbn=true
           , sorting=none
           , style=ieee
           ]{biblatex}
\addbibresource{./../../literature.bib}

% definitions
\def \whatis       {Notes}
\def \title        {Fuubar}

\def \author       {Matthias Schweikardt}

\def \authorMail   {mschweikardt@posteo.de}

\def \authorGithub {mschweikardt}

\def \license      {CC BY-SA 4.0}
\def \licenseUrl   {https://creativecommons.org/licenses/by-sa/4.0/}

\def \date         {nodate}

\def \pdfurl       {https://mschweikardt.github.io/ee-notes/%
\bash[stdout]
IFS=/ 
var=($PWD)
echo ${var[-1]}
\END%
.pdf
}
\def \srcurl       {srcurl}


% Customize footer and header of document
\usepackage{fancyhdr}

% Access last page number
\usepackage{lastpage}

% Access last page number
\usepackage[thinc]{esdiff}

% Physics
\usepackage{physics}

% Comment environment
\usepackage{comment}

% Subcaptions
\usepackage{subcaption}

% Thicker lines in tables
\usepackage{booktabs}

% Indentation in footnote
\makeatletter
\renewcommand\@makefntext[1]{\leftskip=2em\hskip-0.5em\@makefnmark#1}
\makeatother         

% qty with the siunitx definition
\AtBeginDocument{\RenewCommandCopy\qty\SI}

% TikZ compatibility
\pgfplotsset{compat=1.18}


\makeatletter
\pgfmathdeclarefunction{myatan2}{2}{%
\begingroup%
  \pgfmathfloattofixed{#1}\edef\tempa{\pgfmathresult}%
  \pgfmathfloattofixed{#2}%
  \pgfkeys{pgf/fpu=false}%
  \pgfmathparse{atan2(\tempa,\pgfmathresult)}\pgfkeys{/pgf/fpu}%
  \pgfmathfloatparsenumber{\pgfmathresult}%
  \pgfmath@smuggleone\pgfmathresult%
\endgroup
}
\makeatother

\usepackage{tabularx}
\usepackage{amsmath}
\usepackage{amssymb}
\usepackage{nicefrac}

\usepackage{pdflscape}

\usepackage{upgreek}

\usepackage{bashful}

% No intendation
\setlength\parindent{0pt}

\usepackage{hyperref}

\usepackage{siunitx}
\sisetup{
  per-mode=fraction,
  fraction-function=\tfrac
}

\usepackage{listings}
  \lstset{
    basicstyle=\ttfamily,
    escapeinside=||,
    xleftmargin=1cm
  }

\usepackage{float}

\usepackage{longtable}

\usepackage{multirow}

\usepackage{tikz}
  \usetikzlibrary{patterns}
  \usetikzlibrary{arrows.meta}
  \usetikzlibrary{shapes.misc}
  \usetikzlibrary{calc}

\usepackage{pgfplots}

\usepackage{cleveref}
\crefmultiformat{equation}{(#2#1#3)}{ and~(#2#1#3)}{, (#2#1#3)}{ and~(#2#1#3)}


\usepackage{acronym}
\usepackage[acronym,nonumberlist]{glossaries}
\glsdisablehyper
\makeglossaries
\newacronym{spice}{SPICE}{Simulation Program with Integrated Circuit Emphasis}
\newacronym{lef}{LEF}{Library Exchange Format}
\newacronym{dft}{DFT}{Discrete Fourier Transform}
\newacronym{dtft}{DTFT}{Discrete-Time Fourier Transform}
\newacronym{fft}{FFT}{Fast Fourier Transform}
\newacronym{mosfet}{MOSFET}{Metal–Oxide–Semiconductor Field-Effect Transistor}
\newacronym{clm}{CLM}{Channel Length Modulation}
\newacronym{de}{DE}{differential equation}
\newacronym{soi}{SOI}{silicon-on-insulator}
\newacronym{ldo}{LDO}{low-dropout regulator}
\newacronym{ota}{OTA}{operational-transconductance amplifier}
\newacronym{ofa}{OFA}{operational-floating amplifier}

% literature
\usepackage[ backend=biber
           , isbn=true
           , sorting=none
           , style=ieee
           ]{biblatex}
\addbibresource{./../../literature.bib}

% definitions
\def \whatis       {Notes}
\def \title        {Fuubar}

\def \author       {Matthias Schweikardt}

\def \authorMail   {mschweikardt@posteo.de}

\def \authorGithub {mschweikardt}

\def \license      {CC BY-SA 4.0}
\def \licenseUrl   {https://creativecommons.org/licenses/by-sa/4.0/}

\def \date         {nodate}

\def \pdfurl       {https://mschweikardt.github.io/ee-notes/%
\bash[stdout]
IFS=/ 
var=($PWD)
echo ${var[-1]}
\END%
.pdf
}
\def \srcurl       {srcurl}


% Customize footer and header of document
\usepackage{fancyhdr}

% Access last page number
\usepackage{lastpage}

% Access last page number
\usepackage[thinc]{esdiff}

% Physics
\usepackage{physics}

% Comment environment
\usepackage{comment}

% Subcaptions
\usepackage{subcaption}

% Thicker lines in tables
\usepackage{booktabs}

% Indentation in footnote
\makeatletter
\renewcommand\@makefntext[1]{\leftskip=2em\hskip-0.5em\@makefnmark#1}
\makeatother         

% qty with the siunitx definition
\AtBeginDocument{\RenewCommandCopy\qty\SI}

% TikZ compatibility
\pgfplotsset{compat=1.18}


\makeatletter
\pgfmathdeclarefunction{myatan2}{2}{%
\begingroup%
  \pgfmathfloattofixed{#1}\edef\tempa{\pgfmathresult}%
  \pgfmathfloattofixed{#2}%
  \pgfkeys{pgf/fpu=false}%
  \pgfmathparse{atan2(\tempa,\pgfmathresult)}\pgfkeys{/pgf/fpu}%
  \pgfmathfloatparsenumber{\pgfmathresult}%
  \pgfmath@smuggleone\pgfmathresult%
\endgroup
}
\makeatother

\usepackage{tabularx}

\def \title  {Current Mirrors}
\def \date   {July 12, 2025}

\def \pdfurl {https://mschweikardt.github.io/ee-notes/current-mirrors.pdf}
\def \srcurl {https://github.com/mschweikardt/ee-notes/tree/main/notes/current-mirrors}

\usepackage[scale=5]{draftwatermark}

\begin{document}

\notetitle

This file: focus on MOS current mirrors.

\section{Characteristics}

\begin{itemize}
  \item biasing/reference current $I_{\mathrm{R}}$
  
  \item number of outputs
    \begin{itemize}
      \item[-] single output with current $I_{\mathrm{O}}$
      \item[-] multiple outputs with currents $I_{\mathrm{O,1}}$, $I_{\mathrm{O,2}}$, ...
    \end{itemize}

  \item topology
    \begin{itemize}
      \item[-] classic
      \item[-] degenerated
      \item[-] cascode/stacked 
      \item[-] Wilson 
      \item[-] wide-swing
      \item[-] self-biased wide-swing
    \end{itemize}

  \item  additional biasing (dependent on topology)
    \begin{itemize}
      \item[-] voltages $V_{\mathrm{BB,}i}$ 
      \item[-] currents
    \end{itemize}

  \item minimal output voltage $V_{\mathrm{O,min}}$ (for a given $\epsilon$)

  \item output resistance $r_{\mathrm{O}}$

  \item type
    \begin{itemize}
      \item[-] active: currents change with time
      \item[-] passive: currents constant over time (ramp-up of supply omitted)  
    \end{itemize}

  \item bandwidth

  \item area (sum of width times length of all transistors in the current mirror)

  \item mismatch (variation between $I_{\mathrm{R}}$ and $I_{\mathrm{O,}i}$)
    \begin{itemize}
      \item[-] systematic 
        (evaluated for a given lower bound 
          $V_{\mathrm{O,min}}$ and upper bound $V_{\mathrm{O,max}}$)
      \item[-] statistical (evaluated at $V_{\mathrm{O}}=V_{\mathrm{R}}$)
    \end{itemize}
  \item noise
\end{itemize}

\section{Topologies}

\subsection{Simple}


\cite[section 4.2.2.2]{gray-anadesic-09}

\begin{figure}[H]
  \centering
  \begin{circuitikz}
    \input{./../../tikzlib/figs/cm-standard-nmos-schematic-a.tex}
  \end{circuitikz}
  \caption{Simple NMOS current mirror with one output}
  \label{fig:simple-nmos-1}
\end{figure}

\begin{figure}[H]
  \centering
  \begin{circuitikz}
    \input{./../../tikzlib/figs/cm-standard-nmos-schematic-b.tex}
  \end{circuitikz}
  \caption{Simple NMOS current mirror with two outputs}
  \label{fig:simple-nmos-2}
\end{figure}


\begin{figure}[H]
  \centering
  \begin{circuitikz}
    \input{./../../tikzlib/figs/cm-standard-pmos-schematic-a.tex}
  \end{circuitikz}
  \caption{Simple PMOS current mirror with one output}
  \label{fig:simple-pmos-1}
\end{figure}

\subsection{Beta Helper}

\cite{payne-spreadsheet-91,smith-spreadsheet-91,stockstad-rropamp-02}

Not used in simple MOS current mirrors to reduce the systematic gain 
error \cite[section 4.2.3.2]{gray-anadesic-09}.
Can increase the bandwidth of MOS and bipolar current mirrors.


\subsection{Degenerated}

\cite[section 3.5]{johnsmartin-aicd-12}

Source degeneration is rarely used in MOS current mirrors
\cite[section 4.2.4.2]{gray-anadesic-09}.

\subsection{Cascode/Stacked}

\cite[section 3.6]{johnsmartin-aicd-12}

\begin{figure}[H]
  \centering
  \begin{circuitikz}
    \input{./../../tikzlib/figs/cm-cascode-nmos-schematic-a.tex}
  \end{circuitikz}
  \caption{Cascode NMOS current mirror with one output}
  \label{fig:casc-nmos-1}
\end{figure}

\subsection{Self-Cascode}
\cite{self-cascode}

\subsection{Regulated Cascode}

\cite{sackinger-rgc-90}
\cite{coban-rrop-94}
\cite{helfenstein-gainboostop-95}

\subsection{Wilson}

\begin{figure}[H]
  \centering
  \begin{circuitikz}
    \input{./../../tikzlib/figs/cm-wilson-nmos-schematic-a.tex}    
  \end{circuitikz}
  \caption{Wilson NMOS current mirror with one output}
  \label{fig:classic-nmos-1}
\end{figure}



\subsection{Wide-Swing}

\begin{figure}[H]
  \centering
  \begin{circuitikz}
    \input{./../../tikzlib/figs/cm-wide-swing-cascode-nmos-schematic-a.tex}
  \end{circuitikz}
  \caption{Wide-Swing NMOS current mirror with one output}
  \label{fig:ws-nmos-1}
\end{figure}


\subsection{Self-Biased Wide-Swing}

\begin{figure}[H]
  \centering
  \begin{circuitikz}
    \input{./../../tikzlib/figs/cm-self-biased-wide-swing-cascode-nmos-schematic-a.tex}
  \end{circuitikz}
  \caption{Self-Biased Wide-Swing NMOS current mirror with one output}
  \label{fig:sb-ws-nmos-1}
\end{figure}

\subsection{Sooch}

\cite[p. 269]{gray-anadesic-09}
\cite{sooch-casccurrmirr-85}


\subsection{Ian}

\cite{tesu-wscasccurrmirr-09}

\subsection{High-Ratio}
\cite{galupmontoro-serparmosfet-94}
\cite[section 7.2.2]{erckert-aic-22}
\cite{fiorelli-serparmosfet-04}
\cite{arnaud-subnsota-06}

\subsection{Frequency-Dependent}
\cite{itakura-fdcm-96}

\section{Testbench}

\begin{figure}[H]
  \centering
  \begin{circuitikz}
    \input{./../../tikzlib/figs/cm-tb-schematic-a.tex}
  \end{circuitikz}
  \caption{Current mirror testbench}
  \label{fig:testbench}
\end{figure}

Input resistance
\begin{equation}
R_{\mathrm{I}} = \frac{V_{\mathrm{R}}}{I_{\mathrm{R}}}.
\end{equation}

Input admittance
\begin{equation}
\underline{Y}_{\mathrm{I}} = g_{\mathrm{I}} + j \omega C_{\mathrm{I}}
\end{equation}
Output admittance
\begin{equation}
\underline{Y}_{\mathrm{O}} = g_{\mathrm{O}} + j \omega C_{\mathrm{O}}
\end{equation}
evaluated at $V_{\mathrm{O}}=V_{\mathrm{R}}$.



The minimal output voltage of the current mirror is
\begin{equation}
V_{\mathrm{O,min}} : I_{\mathrm{O}}\left(V_{\mathrm{O,min}}\right) = \epsilon \cdot I_{\mathrm{R}}\left(V_{\mathrm{R}}\right),
\end{equation}
with $\epsilon \in (0,1)$.



\printbibliography

\end{document}