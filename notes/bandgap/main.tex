\documentclass{article}[11pt]

\usepackage{amsmath}
\usepackage{amssymb}
\usepackage{nicefrac}

\usepackage{pdflscape}

\usepackage{upgreek}

\usepackage{bashful}

% No intendation
\setlength\parindent{0pt}

\usepackage{hyperref}

\usepackage{siunitx}
\sisetup{
  per-mode=fraction,
  fraction-function=\tfrac
}

\usepackage{listings}
  \lstset{
    basicstyle=\ttfamily,
    escapeinside=||,
    xleftmargin=1cm
  }

\usepackage{float}

\usepackage{longtable}

\usepackage{multirow}

\usepackage{tikz}
  \usetikzlibrary{patterns}
  \usetikzlibrary{arrows.meta}
  \usetikzlibrary{shapes.misc}
  \usetikzlibrary{calc}

\usepackage{pgfplots}

\usepackage{cleveref}
\crefmultiformat{equation}{(#2#1#3)}{ and~(#2#1#3)}{, (#2#1#3)}{ and~(#2#1#3)}


\usepackage{acronym}
\usepackage[acronym,nonumberlist]{glossaries}
\glsdisablehyper
\makeglossaries
\newacronym{spice}{SPICE}{Simulation Program with Integrated Circuit Emphasis}
\newacronym{lef}{LEF}{Library Exchange Format}
\newacronym{dft}{DFT}{Discrete Fourier Transform}
\newacronym{dtft}{DTFT}{Discrete-Time Fourier Transform}
\newacronym{fft}{FFT}{Fast Fourier Transform}
\newacronym{mosfet}{MOSFET}{Metal–Oxide–Semiconductor Field-Effect Transistor}
\newacronym{clm}{CLM}{Channel Length Modulation}
\newacronym{de}{DE}{differential equation}
\newacronym{soi}{SOI}{silicon-on-insulator}
\newacronym{ldo}{LDO}{low-dropout regulator}
\newacronym{ota}{OTA}{operational-transconductance amplifier}
\newacronym{ofa}{OFA}{operational-floating amplifier}

% literature
\usepackage[ backend=biber
           , isbn=true
           , sorting=none
           , style=ieee
           ]{biblatex}
\addbibresource{./../../literature.bib}

% definitions
\def \whatis       {Notes}
\def \title        {Fuubar}

\def \author       {Matthias Schweikardt}

\def \authorMail   {mschweikardt@posteo.de}

\def \authorGithub {mschweikardt}

\def \license      {CC BY-SA 4.0}
\def \licenseUrl   {https://creativecommons.org/licenses/by-sa/4.0/}

\def \date         {nodate}

\def \pdfurl       {https://mschweikardt.github.io/ee-notes/%
\bash[stdout]
IFS=/ 
var=($PWD)
echo ${var[-1]}
\END%
.pdf
}
\def \srcurl       {srcurl}


% Customize footer and header of document
\usepackage{fancyhdr}

% Access last page number
\usepackage{lastpage}

% Access last page number
\usepackage[thinc]{esdiff}

% Physics
\usepackage{physics}

% Comment environment
\usepackage{comment}

% Subcaptions
\usepackage{subcaption}

% Thicker lines in tables
\usepackage{booktabs}

% Indentation in footnote
\makeatletter
\renewcommand\@makefntext[1]{\leftskip=2em\hskip-0.5em\@makefnmark#1}
\makeatother         

% qty with the siunitx definition
\AtBeginDocument{\RenewCommandCopy\qty\SI}

% TikZ compatibility
\pgfplotsset{compat=1.18}


\makeatletter
\pgfmathdeclarefunction{myatan2}{2}{%
\begingroup%
  \pgfmathfloattofixed{#1}\edef\tempa{\pgfmathresult}%
  \pgfmathfloattofixed{#2}%
  \pgfkeys{pgf/fpu=false}%
  \pgfmathparse{atan2(\tempa,\pgfmathresult)}\pgfkeys{/pgf/fpu}%
  \pgfmathfloatparsenumber{\pgfmathresult}%
  \pgfmath@smuggleone\pgfmathresult%
\endgroup
}
\makeatother

\usepackage{tabularx}
\usepackage{amsmath}
\usepackage{amssymb}
\usepackage{nicefrac}

\usepackage{pdflscape}

\usepackage{upgreek}

\usepackage{bashful}

% No intendation
\setlength\parindent{0pt}

\usepackage{hyperref}

\usepackage{siunitx}
\sisetup{
  per-mode=fraction,
  fraction-function=\tfrac
}

\usepackage{listings}
  \lstset{
    basicstyle=\ttfamily,
    escapeinside=||,
    xleftmargin=1cm
  }

\usepackage{float}

\usepackage{longtable}

\usepackage{multirow}

\usepackage{tikz}
  \usetikzlibrary{patterns}
  \usetikzlibrary{arrows.meta}
  \usetikzlibrary{shapes.misc}
  \usetikzlibrary{calc}

\usepackage{pgfplots}

\usepackage{cleveref}
\crefmultiformat{equation}{(#2#1#3)}{ and~(#2#1#3)}{, (#2#1#3)}{ and~(#2#1#3)}


\usepackage{acronym}
\usepackage[acronym,nonumberlist]{glossaries}
\glsdisablehyper
\makeglossaries
\newacronym{spice}{SPICE}{Simulation Program with Integrated Circuit Emphasis}
\newacronym{lef}{LEF}{Library Exchange Format}
\newacronym{dft}{DFT}{Discrete Fourier Transform}
\newacronym{dtft}{DTFT}{Discrete-Time Fourier Transform}
\newacronym{fft}{FFT}{Fast Fourier Transform}
\newacronym{mosfet}{MOSFET}{Metal–Oxide–Semiconductor Field-Effect Transistor}
\newacronym{clm}{CLM}{Channel Length Modulation}
\newacronym{de}{DE}{differential equation}
\newacronym{soi}{SOI}{silicon-on-insulator}
\newacronym{ldo}{LDO}{low-dropout regulator}
\newacronym{ota}{OTA}{operational-transconductance amplifier}
\newacronym{ofa}{OFA}{operational-floating amplifier}

% literature
\usepackage[ backend=biber
           , isbn=true
           , sorting=none
           , style=ieee
           ]{biblatex}
\addbibresource{./../../literature.bib}

% definitions
\def \whatis       {Notes}
\def \title        {Fuubar}

\def \author       {Matthias Schweikardt}

\def \authorMail   {mschweikardt@posteo.de}

\def \authorGithub {mschweikardt}

\def \license      {CC BY-SA 4.0}
\def \licenseUrl   {https://creativecommons.org/licenses/by-sa/4.0/}

\def \date         {nodate}

\def \pdfurl       {https://mschweikardt.github.io/ee-notes/%
\bash[stdout]
IFS=/ 
var=($PWD)
echo ${var[-1]}
\END%
.pdf
}
\def \srcurl       {srcurl}


% Customize footer and header of document
\usepackage{fancyhdr}

% Access last page number
\usepackage{lastpage}

% Access last page number
\usepackage[thinc]{esdiff}

% Physics
\usepackage{physics}

% Comment environment
\usepackage{comment}

% Subcaptions
\usepackage{subcaption}

% Thicker lines in tables
\usepackage{booktabs}

% Indentation in footnote
\makeatletter
\renewcommand\@makefntext[1]{\leftskip=2em\hskip-0.5em\@makefnmark#1}
\makeatother         

% qty with the siunitx definition
\AtBeginDocument{\RenewCommandCopy\qty\SI}

% TikZ compatibility
\pgfplotsset{compat=1.18}


\makeatletter
\pgfmathdeclarefunction{myatan2}{2}{%
\begingroup%
  \pgfmathfloattofixed{#1}\edef\tempa{\pgfmathresult}%
  \pgfmathfloattofixed{#2}%
  \pgfkeys{pgf/fpu=false}%
  \pgfmathparse{atan2(\tempa,\pgfmathresult)}\pgfkeys{/pgf/fpu}%
  \pgfmathfloatparsenumber{\pgfmathresult}%
  \pgfmath@smuggleone\pgfmathresult%
\endgroup
}
\makeatother

\usepackage{tabularx}


\def \title  {Bandgap References}
\def \date   {May 29, 2025}

\def \pdfurl {https://mschweikardt.github.io/ee-notes/bandgap.pdf}
\def \srcurl {https://github.com/mschweikardt/ee-notes/tree/main/notes/bandgap}

\usepackage[scale=5]{draftwatermark}

\begin{document}

\notetitle


\cite{widlar-bandgap-71}
\cite{kuijk-bandgap-73}
\cite{brokaw-bandgap-74}
\cite{vittoz-bandgap-79}

\cite{razavi-bandgap-16}
\cite{razavi-bandgap-21}

\cite{rinmora-volref-02}
\cite{ivanov-bandgap-12}
\cite{kok-cmosvoltref-13}
\cite{ma-bandgap-14}
\cite{mohamed-bandgap-19}
\cite{huang-bandgap-21}

\cite{moisello-bandgap-24}


\begin{figure}[H]
  \centering
  \begin{circuitikz}
    \input{./../../tikzlib/figs/bandgap-cmos-schematic-a.tex}
  \end{circuitikz}
  \caption{Bandgap Core}
  \label{fig:badgap-core}
\end{figure}

The V-I characteristic of an PNP transistor is approximated with
\begin{equation}
 I_E \approx I_{\mathrm{S}} \cdot e^{\frac{V_\mathrm{EB}}{V_\mathrm{T}}},
\end{equation}
with the \textit{emitter~current} $I_E$,
the \textit{reverse~saturation~current} $I_{\mathrm{S}}$, the
\textit{Emitter-Base Voltage} $V_\mathrm{EB}$ an the \textit{thermal~voltage}
\begin{equation}\label{eq:vt}
V_\mathrm{T}=\frac{kT}{e},
\end{equation}
with the \textit{absolute~temperature} $T$ in K,
the \textit{Boltzmann~constant}~$k=\SI{1.38e-23}{\joule\per\kelvin}$
and \textit{electron~charge}~$e=\SI{1.602e-19}{\coulomb}$.
For the configuration in Fig.~\ref{fig:badgap-core}, this results in the
following equations
\footnote{The emitter area of the right PNP has the size N of the left PNP.}:
\begin{equation}\label{eq:i1}
 I_1 = N \cdot I_{\mathrm{S}} \cdot e^{\frac{V_\mathrm{EB1}}{V_\mathrm{T}}}
\end{equation}
\begin{equation}\label{eq:i2}
 I_2 = I_{\mathrm{S}} \cdot e^{\frac{V_\mathrm{EB2}}{V_\mathrm{T}}}
\end{equation}
The PMOS current mirrors forces the two currents to be identical, i.e.
\begin{equation}\label{eq:curr}
 I_1 \stackrel{!}{=} I_2.
\end{equation}
Using \eqref{eq:i1}, \eqref{eq:i2} and \eqref{eq:curr} one can calculate
\begin{equation}
V_\mathrm{EB2}-V_\mathrm{EB1} = V_\mathrm{T} \cdot \ln(N).
\end{equation}

We assume that the opamp is ideal, thus the input differential voltage
$V_{\mathrm{D}}$  is equal to zero.
As a consequence
\begin{equation}
V_{\mathrm{R1}} = V_\mathrm{EB2}-V_\mathrm{EB1}.
\end{equation}
and
\begin{equation}\label{eq:i1-calc}
I_1 = \frac{V_{\mathrm{R1}}}{R_1} = \frac{V_\mathrm{T} \cdot \ln(N)}{R_1}.
\end{equation}
The output voltage of the bandgap $V_{\mathrm{BG}}$ is
\begin{equation}\label{eq:vbg1}
V_{\mathrm{BG}} = V_{\mathrm{EB1}} + I_1 \left(R_1+R_2\right).
\end{equation}
Inserting \eqref{eq:i1-calc}, \eqref{eq:vt} and \eqref{eq:vbg1} results in
\begin{equation}\label{eq:vbg1}
V_{\mathrm{BG}} = V_{\mathrm{EB1}} + 
                \frac{kT}{e} \cdot \ln(N) \cdot \left(1+\frac{R_2}{R_1}\right).
\end{equation}
By deriving \eqref{eq:vbg1} to the temperature results in
\footnote{The ratio $\nicefrac{R_2}{R_1}$ is temperature-independent.}
\begin{equation}
\frac{\mathrm{d}V_{\mathrm{BG}}}{\mathrm{d}T}
= \frac{\mathrm{d}V_{\mathrm{EB1}}}{\mathrm{d}T}
+ \frac{k}{e} \cdot \ln(N) \cdot \left(1+\frac{R_2}{R_1}\right).
\end{equation}

The derivate of the bandgap voltage at a derided temperature $T_0$
must be zero, i.e.
\begin{equation}
\frac{\mathrm{d}V_{\mathrm{BG}}}{\mathrm{d}T}\bigg|_{T=T_0} = 0.
\end{equation}

As a result, the equation 
\begin{equation}
R_2 + R_1\left(1+\frac{e}{k \ln(N)} \cdot \frac{\mathrm{d}V_{\mathrm{EB1}}}{\mathrm{d}T}\bigg|_{T=T_0}\right) = 0.
\end{equation}
must be fulfilled.

\printbibliography
\end{document}