\documentclass{article}[11pt]

\usepackage{amsmath}
\usepackage{amssymb}
\usepackage{nicefrac}

\usepackage{pdflscape}

\usepackage{upgreek}

\usepackage{bashful}

% No intendation
\setlength\parindent{0pt}

\usepackage{hyperref}

\usepackage{siunitx}
\sisetup{
  per-mode=fraction,
  fraction-function=\tfrac
}

\usepackage{listings}
  \lstset{
    basicstyle=\ttfamily,
    escapeinside=||,
    xleftmargin=1cm
  }

\usepackage{float}

\usepackage{longtable}

\usepackage{multirow}

\usepackage{tikz}
  \usetikzlibrary{patterns}
  \usetikzlibrary{arrows.meta}
  \usetikzlibrary{shapes.misc}
  \usetikzlibrary{calc}

\usepackage{pgfplots}

\usepackage{cleveref}
\crefmultiformat{equation}{(#2#1#3)}{ and~(#2#1#3)}{, (#2#1#3)}{ and~(#2#1#3)}


\usepackage{acronym}
\usepackage[acronym,nonumberlist]{glossaries}
\glsdisablehyper
\makeglossaries
\newacronym{spice}{SPICE}{Simulation Program with Integrated Circuit Emphasis}
\newacronym{lef}{LEF}{Library Exchange Format}
\newacronym{dft}{DFT}{Discrete Fourier Transform}
\newacronym{dtft}{DTFT}{Discrete-Time Fourier Transform}
\newacronym{fft}{FFT}{Fast Fourier Transform}
\newacronym{mosfet}{MOSFET}{Metal–Oxide–Semiconductor Field-Effect Transistor}
\newacronym{clm}{CLM}{Channel Length Modulation}
\newacronym{de}{DE}{differential equation}
\newacronym{soi}{SOI}{silicon-on-insulator}
\newacronym{ldo}{LDO}{low-dropout regulator}
\newacronym{ota}{OTA}{operational-transconductance amplifier}
\newacronym{ofa}{OFA}{operational-floating amplifier}

% literature
\usepackage[ backend=biber
           , isbn=true
           , sorting=none
           , style=ieee
           ]{biblatex}
\addbibresource{./../../literature.bib}

% definitions
\def \whatis       {Notes}
\def \title        {Fuubar}

\def \author       {Matthias Schweikardt}

\def \authorMail   {mschweikardt@posteo.de}

\def \authorGithub {mschweikardt}

\def \license      {CC BY-SA 4.0}
\def \licenseUrl   {https://creativecommons.org/licenses/by-sa/4.0/}

\def \date         {nodate}

\def \pdfurl       {https://mschweikardt.github.io/ee-notes/%
\bash[stdout]
IFS=/ 
var=($PWD)
echo ${var[-1]}
\END%
.pdf
}
\def \srcurl       {srcurl}


% Customize footer and header of document
\usepackage{fancyhdr}

% Access last page number
\usepackage{lastpage}

% Access last page number
\usepackage[thinc]{esdiff}

% Physics
\usepackage{physics}

% Comment environment
\usepackage{comment}

% Subcaptions
\usepackage{subcaption}

% Thicker lines in tables
\usepackage{booktabs}

% Indentation in footnote
\makeatletter
\renewcommand\@makefntext[1]{\leftskip=2em\hskip-0.5em\@makefnmark#1}
\makeatother         

% qty with the siunitx definition
\AtBeginDocument{\RenewCommandCopy\qty\SI}

% TikZ compatibility
\pgfplotsset{compat=1.18}


\makeatletter
\pgfmathdeclarefunction{myatan2}{2}{%
\begingroup%
  \pgfmathfloattofixed{#1}\edef\tempa{\pgfmathresult}%
  \pgfmathfloattofixed{#2}%
  \pgfkeys{pgf/fpu=false}%
  \pgfmathparse{atan2(\tempa,\pgfmathresult)}\pgfkeys{/pgf/fpu}%
  \pgfmathfloatparsenumber{\pgfmathresult}%
  \pgfmath@smuggleone\pgfmathresult%
\endgroup
}
\makeatother

\usepackage{tabularx}
\usepackage{amsmath}
\usepackage{amssymb}
\usepackage{nicefrac}

\usepackage{pdflscape}

\usepackage{upgreek}

\usepackage{bashful}

% No intendation
\setlength\parindent{0pt}

\usepackage{hyperref}

\usepackage{siunitx}
\sisetup{
  per-mode=fraction,
  fraction-function=\tfrac
}

\usepackage{listings}
  \lstset{
    basicstyle=\ttfamily,
    escapeinside=||,
    xleftmargin=1cm
  }

\usepackage{float}

\usepackage{longtable}

\usepackage{multirow}

\usepackage{tikz}
  \usetikzlibrary{patterns}
  \usetikzlibrary{arrows.meta}
  \usetikzlibrary{shapes.misc}
  \usetikzlibrary{calc}

\usepackage{pgfplots}

\usepackage{cleveref}
\crefmultiformat{equation}{(#2#1#3)}{ and~(#2#1#3)}{, (#2#1#3)}{ and~(#2#1#3)}


\usepackage{acronym}
\usepackage[acronym,nonumberlist]{glossaries}
\glsdisablehyper
\makeglossaries
\newacronym{spice}{SPICE}{Simulation Program with Integrated Circuit Emphasis}
\newacronym{lef}{LEF}{Library Exchange Format}
\newacronym{dft}{DFT}{Discrete Fourier Transform}
\newacronym{dtft}{DTFT}{Discrete-Time Fourier Transform}
\newacronym{fft}{FFT}{Fast Fourier Transform}
\newacronym{mosfet}{MOSFET}{Metal–Oxide–Semiconductor Field-Effect Transistor}
\newacronym{clm}{CLM}{Channel Length Modulation}
\newacronym{de}{DE}{differential equation}
\newacronym{soi}{SOI}{silicon-on-insulator}
\newacronym{ldo}{LDO}{low-dropout regulator}
\newacronym{ota}{OTA}{operational-transconductance amplifier}
\newacronym{ofa}{OFA}{operational-floating amplifier}

% literature
\usepackage[ backend=biber
           , isbn=true
           , sorting=none
           , style=ieee
           ]{biblatex}
\addbibresource{./../../literature.bib}

% definitions
\def \whatis       {Notes}
\def \title        {Fuubar}

\def \author       {Matthias Schweikardt}

\def \authorMail   {mschweikardt@posteo.de}

\def \authorGithub {mschweikardt}

\def \license      {CC BY-SA 4.0}
\def \licenseUrl   {https://creativecommons.org/licenses/by-sa/4.0/}

\def \date         {nodate}

\def \pdfurl       {https://mschweikardt.github.io/ee-notes/%
\bash[stdout]
IFS=/ 
var=($PWD)
echo ${var[-1]}
\END%
.pdf
}
\def \srcurl       {srcurl}


% Customize footer and header of document
\usepackage{fancyhdr}

% Access last page number
\usepackage{lastpage}

% Access last page number
\usepackage[thinc]{esdiff}

% Physics
\usepackage{physics}

% Comment environment
\usepackage{comment}

% Subcaptions
\usepackage{subcaption}

% Thicker lines in tables
\usepackage{booktabs}

% Indentation in footnote
\makeatletter
\renewcommand\@makefntext[1]{\leftskip=2em\hskip-0.5em\@makefnmark#1}
\makeatother         

% qty with the siunitx definition
\AtBeginDocument{\RenewCommandCopy\qty\SI}

% TikZ compatibility
\pgfplotsset{compat=1.18}


\makeatletter
\pgfmathdeclarefunction{myatan2}{2}{%
\begingroup%
  \pgfmathfloattofixed{#1}\edef\tempa{\pgfmathresult}%
  \pgfmathfloattofixed{#2}%
  \pgfkeys{pgf/fpu=false}%
  \pgfmathparse{atan2(\tempa,\pgfmathresult)}\pgfkeys{/pgf/fpu}%
  \pgfmathfloatparsenumber{\pgfmathresult}%
  \pgfmath@smuggleone\pgfmathresult%
\endgroup
}
\makeatother

\usepackage{tabularx}
\include{./../../tikzlib/figs/ekv-ic-vs-vgseff.tex}


\def \title  {EKV MOSFET Model}
\def \date   {May 9, 2025}

\def \pdfurl {https://mschweikardt.github.io/ee-notes/ekv.pdf}
\def \srcurl {https://github.com/mschweikardt/ee-notes/tree/main/notes/ekv}

\usepackage[scale=5]{draftwatermark}

\begin{document}

\notetitle

EKV%
\footnote{Named by their inventors Enz, Krummenacher and Vittoz}
is an analytical \gls{mosfet} model \cite{enz-ekv-90}.
The drain current of the transistor is
\begin{equation}
I_{\mathrm{D}} = I_{\mathrm{S}} \left(\mathrm{IC}_{\mathrm{F}}-\mathrm{IC}_{\mathrm{R}}\right),
\end{equation}
with the specific current $I_{\mathrm{S}}$ (unit A), the forward 
inversion coefficient 
$\mathrm{IC}_{\mathrm{F}}(V_{\mathrm{G}}, V_{\mathrm{S}})$ and 
the backward inversion coefficient 
$\mathrm{IC}_{\mathrm{R}}(V_{\mathrm{G}}, V_{\mathrm{D}})$.
The inversion coefficient $\mathrm{IC}$ is a proxy for the inversion level of 
the channel of the \gls{mosfet}.

\medskip

The specific current is given by
\begin{equation}
  I_{\mathrm{S}} = 2 n V_{\mathrm{T}}^2 K' \frac{W}{L},
\end{equation}
with the temperature voltage $V_{\mathrm{T}}$%
\footnote{$V_{\mathrm{T}}=\frac{k_{\mathrm{B}} T }{e}$, with the 
Boltzmann constant $k_{\mathrm{B}}$, the absolute temperature $T$
in \si{\kelvin} and the charge of an electron $e$.}
,
the subthreshold slope $n$ and the transconductance coefficient $K'$ 
(conf. \cite{mosfet-square-law}).
Here, we will only investigate the forward region, so 
$\mathrm{IC}_{\mathrm{R}} \rightarrow 0$.

\medskip

The forward inversion coefficient is given by 
(subscript $\mathrm{F}$ omitted)

\begin{equation}\label{eq:ic}
  \mathrm{IC} = \ln\left(1+\exp\left\{\frac{V_{\mathrm{GS}}-V_{\mathrm{TH}}}{2 n V_{\mathrm{T}}}\right\}\right)^2.
\end{equation}


When $V_{\mathrm{GS}}-V_{\mathrm{TH}} \gg 0$, we can approximate 
it with

\begin{equation}\label{eq:ic-quad}
  \mathrm{IC}_{\mathrm{quad}} = \left(\frac{V_{\mathrm{GS}}-V_{\mathrm{TH}}}{2 n V_{\mathrm{T}}}\right)^2
\end{equation}

and for $V_{\mathrm{GS}}-V_{\mathrm{TH}} \ll 0$ with

\begin{equation}\label{eq:ic-exp}
  \mathrm{IC}_{\mathrm{exp}} = \exp\left\{\frac{V_{\mathrm{GS}}-V_{\mathrm{TH}}}{n V_{\mathrm{T}}}\right\}.
\end{equation}

$\mathrm{IC}_{\mathrm{quad}}$ is quadratic and is therefore a 
reasonable approximation for the \textit{saturation} region, while 
$\mathrm{IC}_{\mathrm{exp}}$ is exponential and is a good 
approximation for the \textit{subthreshold} region.
Fig. \ref{fig:plot} plots \eqref{eq:ic}, \eqref{eq:ic-quad} and  
\eqref{eq:ic-exp}.

\begin{figure}[h]
  \centering
  \begin{tikzpicture}
    \EkvIcVsVgseff
  \end{tikzpicture}
  \caption{$\mathrm{IC}$, $\mathrm{IC}_{\mathrm{exp}}$ and 
    $\mathrm{IC}_{\mathrm{quad}}$ as a function of 
    $\frac{V_{\mathrm{GS}}-V_{\mathrm{TH}}}{2 n V_{\mathrm{T}}}$ with added 
    inversion regions}
  \label{fig:plot}
\end{figure}

Neither $\mathrm{IC}_{\mathrm{quad}}$ or $\mathrm{IC}_{\mathrm{exp}}$
are sufficient to approximate the behavior of $\mathrm{IC}$
close to $V_{\mathrm{GS}}=V_{\mathrm{TH}}$.

\medskip

Tab.~\ref{tab:ekv-regions} summarizes the individual $\mathrm{IC}$ ranges,
their corresponding inversion level and curvature.

\begin{table}[h]
\centering
\caption{EKV regions}
\begin{tabular}{ccccc}
\toprule
Identifier           & Name                   & $\mathrm{IC}$  & Curvature     & Current             \\ \midrule
SI                   & strong inversion       & $\geq 10$      & quadratic     & drift               \\ 
MI                   & moderate inversion     & $0.1$-$10$     & ?             & drift and diffusion \\ 
WI                   & weak inversion         & $\leq 0.1$     & exponential   & diffusion           \\ \toprule
\end{tabular}
\label{tab:ekv-regions}
\end{table}

\printbibliography

\end{document}