\documentclass{article}[11pt]

% usepackages
\usepackage[ a4paper
           , textwidth  = 16.0cm
           , textheight = 25.0cm
           , headsep    =  0.25cm
           , voffset    =  0.3cm
           , footskip   =  1.25cm
           ]{geometry}

\usepackage{amsmath}
\usepackage{amssymb}
\usepackage{nicefrac}

% No intendation
\setlength\parindent{0pt}

\usepackage{hyperref}

\usepackage{siunitx}
\sisetup{
  per-mode=fraction,
  fraction-function=\tfrac
}

\usepackage{listings}
  \lstset{
    basicstyle=\ttfamily,
    escapeinside=||,
    xleftmargin=1cm
  }

\usepackage{float}

\usepackage{tikz}
  \usetikzlibrary{patterns}
  \usetikzlibrary{arrows.meta}
  \usetikzlibrary{shapes.misc}
  \usetikzlibrary{calc}

\usepackage{pgfplots}

\usepackage{cleveref}
\crefmultiformat{equation}{(#2#1#3)}{ and~(#2#1#3)}{, (#2#1#3)}{ and~(#2#1#3)}


\usepackage{acronym}
\usepackage[acronym,nonumberlist]{glossaries}
\makeglossaries
\newacronym{spice}{SPICE}{Simulation Program with Integrated Circuit Emphasis}
\newacronym{lef}{LEF}{Library Exchange Format}
\newacronym{dft}{DFT}{Discrete Fourier Transform}
\newacronym{dtft}{DTFT}{Discrete-Time Fourier Transform}
\newacronym{fft}{FFT}{Fast Fourier Transform}
\newacronym{mosfet}{MOSFET}{Metal–Oxide–Semiconductor Field-Effect Transistor}
\newacronym{clm}{CLM}{Channel Length Modulation}

% Section and subsection enumeration
\renewcommand{\thesection}{\Roman{section}.} 
\renewcommand{\thesubsection}{\thesection\Alph{subsection}}

% literature
\usepackage[backend=biber, isbn=true, sorting=none]{biblatex}
\addbibresource{./../../literature.bib}

% definitions
\def \whatis       {Notes}
\def \title        {Fuubar}

\def \author       {Matthias Schweikardt}

\def \authorMail   {mschweikardt@posteo.de}

\def \authorGithub {mschweikardt}

\def \license      {CC BY-SA 4.0}
\def \licenseUrl   {https://creativecommons.org/licenses/by-sa/4.0/}

\def \date         {nodate}

\def \pdfurl       {pdfurl}
\def \srcurl       {srcurl}


% Customize footer and header of document
\usepackage{fancyhdr}

% Access last page number
\usepackage{lastpage}

% Access last page number
\usepackage[thinc]{esdiff}

% Physics
\usepackage{physics}

% Comment environment
\usepackage{comment}

% Subcaptions
\usepackage{subcaption}

% Thicker lines in tables
\usepackage{booktabs}


% Defince title of document
\newcommand{\notetitle}{
  \begingroup
  \hypersetup{hidelinks}
  \thispagestyle{notefirst}
  \begin{center}
  \rule{\textwidth}{1pt}\\
  \medskip
  {\it \whatis}\\
  \bigskip
  {\LARGE \textbf{\title}}\\
  \medskip
  {\small \author}\\
  \rule{\textwidth}{0.5pt}\\
  {\small
    \begin{minipage}[t]{0.5\textwidth}
      \begin{tabular}[t]{ p{2.25cm} p{5.75cm}}
        Mail: & \href{mailto:\authorMail}{\tt{\authorMail}} \\
        Github: & \href{https://github.com/\authorGithub}{\tt{\authorGithub}} \\
      \end{tabular}
    \end{minipage}%
    %
    \begin{minipage}[t]{0.5\textwidth}
      \begin{tabular}[t]{ p{2.25cm} p{5.75cm} }
        Date: & \date  \\
        License: & \href{\licenseUrl}{\license}
      \end{tabular}
    \end{minipage}
  }%
  {\small
    \begin{minipage}[t]{\textwidth}
      \begin{tabular}[t]{ p{2.25cm} p{12cm}}
        Latest PDF: & \href{\pdfurl}{\tt{\pdfurl}} \\
        Latest Source: & \href{\srcurl}{\tt{\srcurl}}
      \end{tabular}
    \end{minipage}%
  }
  \bigskip
  \rule{\textwidth}{1pt}
  \end{center}
  \endgroup
}

% Header and footer on first page
\fancypagestyle{notefirst}{
  \fancyhf{}
  \renewcommand{\headrulewidth}{0pt}
  \renewcommand{\footrulewidth}{0pt}

  \fancyfoot[C]{\thepage/\pageref*{LastPage}}
}

% Header and footer on 2nd-last page
\fancypagestyle{noterest}{
  \fancyhf{}
  \renewcommand{\headrulewidth}{0.5pt}
  \renewcommand{\footrulewidth}{0.0pt}

  \fancyhead[L]{\author}
  \fancyhead[C]{\title}
  \fancyhead[R]{\date}

  \fancyfoot[C]{\thepage/\pageref*{LastPage}}
}
\pagestyle{noterest}


% Indentation in footnote
\makeatletter
\renewcommand\@makefntext[1]{\leftskip=2em\hskip-0.5em\@makefnmark#1}
\makeatother

% usepackages
\usepackage[ a4paper
           , textwidth  = 16.0cm
           , textheight = 25.0cm
           , headsep    =  0.25cm
           , voffset    =  0.3cm
           , footskip   =  1.25cm
           ]{geometry}

\usepackage{amsmath}
\usepackage{amssymb}
\usepackage{nicefrac}

% No intendation
\setlength\parindent{0pt}

\usepackage{hyperref}

\usepackage{siunitx}
\sisetup{
  per-mode=fraction,
  fraction-function=\tfrac
}

\usepackage{listings}
  \lstset{
    basicstyle=\ttfamily,
    escapeinside=||,
    xleftmargin=1cm
  }

\usepackage{float}

\usepackage{tikz}
  \usetikzlibrary{patterns}
  \usetikzlibrary{arrows.meta}
  \usetikzlibrary{shapes.misc}
  \usetikzlibrary{calc}

\usepackage{pgfplots}

\usepackage{cleveref}
\crefmultiformat{equation}{(#2#1#3)}{ and~(#2#1#3)}{, (#2#1#3)}{ and~(#2#1#3)}


\usepackage{acronym}
\usepackage[acronym,nonumberlist]{glossaries}
\makeglossaries
\newacronym{spice}{SPICE}{Simulation Program with Integrated Circuit Emphasis}
\newacronym{lef}{LEF}{Library Exchange Format}
\newacronym{dft}{DFT}{Discrete Fourier Transform}
\newacronym{dtft}{DTFT}{Discrete-Time Fourier Transform}
\newacronym{fft}{FFT}{Fast Fourier Transform}
\newacronym{mosfet}{MOSFET}{Metal–Oxide–Semiconductor Field-Effect Transistor}
\newacronym{clm}{CLM}{Channel Length Modulation}

% Section and subsection enumeration
\renewcommand{\thesection}{\Roman{section}.} 
\renewcommand{\thesubsection}{\thesection\Alph{subsection}}

% literature
\usepackage[backend=biber, isbn=true, sorting=none]{biblatex}
\addbibresource{./../../literature.bib}

% definitions
\def \whatis       {Notes}
\def \title        {Fuubar}

\def \author       {Matthias Schweikardt}

\def \authorMail   {mschweikardt@posteo.de}

\def \authorGithub {mschweikardt}

\def \license      {CC BY-SA 4.0}
\def \licenseUrl   {https://creativecommons.org/licenses/by-sa/4.0/}

\def \date         {nodate}

\def \pdfurl       {pdfurl}
\def \srcurl       {srcurl}


% Customize footer and header of document
\usepackage{fancyhdr}

% Access last page number
\usepackage{lastpage}

% Access last page number
\usepackage[thinc]{esdiff}

% Physics
\usepackage{physics}

% Comment environment
\usepackage{comment}

% Subcaptions
\usepackage{subcaption}

% Thicker lines in tables
\usepackage{booktabs}


% Defince title of document
\newcommand{\notetitle}{
  \begingroup
  \hypersetup{hidelinks}
  \thispagestyle{notefirst}
  \begin{center}
  \rule{\textwidth}{1pt}\\
  \medskip
  {\it \whatis}\\
  \bigskip
  {\LARGE \textbf{\title}}\\
  \medskip
  {\small \author}\\
  \rule{\textwidth}{0.5pt}\\
  {\small
    \begin{minipage}[t]{0.5\textwidth}
      \begin{tabular}[t]{ p{2.25cm} p{5.75cm}}
        Mail: & \href{mailto:\authorMail}{\tt{\authorMail}} \\
        Github: & \href{https://github.com/\authorGithub}{\tt{\authorGithub}} \\
      \end{tabular}
    \end{minipage}%
    %
    \begin{minipage}[t]{0.5\textwidth}
      \begin{tabular}[t]{ p{2.25cm} p{5.75cm} }
        Date: & \date  \\
        License: & \href{\licenseUrl}{\license}
      \end{tabular}
    \end{minipage}
  }%
  {\small
    \begin{minipage}[t]{\textwidth}
      \begin{tabular}[t]{ p{2.25cm} p{12cm}}
        Latest PDF: & \href{\pdfurl}{\tt{\pdfurl}} \\
        Latest Source: & \href{\srcurl}{\tt{\srcurl}}
      \end{tabular}
    \end{minipage}%
  }
  \bigskip
  \rule{\textwidth}{1pt}
  \end{center}
  \endgroup
}

% Header and footer on first page
\fancypagestyle{notefirst}{
  \fancyhf{}
  \renewcommand{\headrulewidth}{0pt}
  \renewcommand{\footrulewidth}{0pt}

  \fancyfoot[C]{\thepage/\pageref*{LastPage}}
}

% Header and footer on 2nd-last page
\fancypagestyle{noterest}{
  \fancyhf{}
  \renewcommand{\headrulewidth}{0.5pt}
  \renewcommand{\footrulewidth}{0.0pt}

  \fancyhead[L]{\author}
  \fancyhead[C]{\title}
  \fancyhead[R]{\date}

  \fancyfoot[C]{\thepage/\pageref*{LastPage}}
}
\pagestyle{noterest}


% Indentation in footnote
\makeatletter
\renewcommand\@makefntext[1]{\leftskip=2em\hskip-0.5em\@makefnmark#1}
\makeatother



\def \title  {Operational Transconductance Amplifier}
\def \date   {June 2, 2025}

\def \pdfurl {https://mschweikardt.github.io/ee-notes/ota.pdf}
\def \srcurl {https://github.com/mschweikardt/ee-notes/tree/main/notes/ota}

\usepackage[scale=5]{draftwatermark}

\begin{document}

\notetitle


\begin{circuitikz}
\node[op amp,yscale=-1] at (0,0) {};
\end{circuitikz}

\begin{circuitikz}
\node[gm amp,yscale=-1] at (0,0) {};
\end{circuitikz}

\begin{figure}[H]
  \centering
  \begin{circuitikz}

    \newcommand\height{2.0}
    \draw [] (0, \height/2)                       coordinate (vip) 
       to [short,o-] ++ (1,0)
       to [C,l_=$C_{\mathrm{in}}$] ++ (0,-\height) coordinate (x)
       to [short,-o]                  (x-|vip)     coordinate (vin);

    \draw (6.75, \height/2) coordinate (vop)  
         to [short,o-] ++ (-1.25,0) coordinate (cop)      
         to [short,-] ++ (-1.25,0) coordinate (rop)  
         to [short,*-] ++ (-2.25,0)
         to [ american controlled current source
            , l=$G_{\mathrm{m}} \cdot \underline{V}_{\mathrm{I}}$
            , -*
            ] ++ (0,-\height) coordinate (gnd)
         to [short,-*] (gnd-|cop) coordinate (con)  
         to [short,-*] (gnd-|rop) coordinate (ron)         
         to [short,-o] (gnd-|vop) coordinate (von);

    \draw (rop) to [R,l=$r_{\mathrm{out}}$]  (ron);
    \draw (cop) to [C,l=$C_{\mathrm{out}}$]  (con);
    \node[ground] at (gnd) {};

    \draw (vop) to [open,o-] ++ (1,0) coordinate (x)
                to [short,-] ++ (0.5,0) coordinate (clp)
                to [C,l=$C_{\mathrm{L}}$]  (clp|-ron)
                to [short,-] ++ (-0.5,0) coordinate (y);

    \draw[densely dotted] (x) --++ (-0.85,0);
    \draw[densely dotted] (y) --++ (-0.85,0);

    \path [voltarrow] (vip) edge node [midway,left,inner sep=2pt] 
      {$\underline{V}_{\mathrm{I}}$} (vin);
    \path [voltarrow] (vop) edge node [midway,right,inner sep=2pt] 
      {$\underline{V}_{\mathrm{O}}$} (von);

    \node[anchor=east] at (vip) {INP};
    \node[anchor=east] at (vin) {INN};
    \node[anchor=south] at (vop) {O};  
  \end{circuitikz}
  \caption{First-order model of an \gls{ota}, adapted from \cite{johnsmartin-aicd-12}}
  \label{fig:badgap-core}
\end{figure}

\section{Definition}

\cite[p. 261]{allen-cmosancirdes-12} says that an \gls{ota} has a high output 
resistance (unbuffered) in contrast to low output resistance amplifiers 
(voltage operational amplifier, buffered).

\medskip

\cite[section 24.3]{baker-cmoscircdesnsim-10} says that an \gls{ota} is an amplifier 
where all nodes are low impedance except the input and output nodes.
An \gls{ota} without buffer can only drive capacitive loads.
Any resistive load will reduce the gain significantly.

\medskip

\cite[section 6.4]{johnsmartin-aicd-12} says and \gls{ota}
\begin{itemize}
  \item designed to drive only capacitive loads
  \item With feedback, they can also drive some resistive load.
  \item not necessary to use a voltage buffer to obtain a low output impedance for the opamp
  \item only a single high-impedance node at the output
  \item admittance seen at all other nodes is relatively low -> speed is maximized
  \item reduced voltage signals at all nodes other than the output node; however, the current signals in the various transistors can be quite large
  \item compensation is usually achieved by the load capacitance
  \item load capacitance gets larger, the opamp usually becomes more stable but also slower.
\end{itemize}

\medskip

\cite[055 ff.]{sansen-anadesess-06}
\begin{itemize}
  \item output is high impedance
  \item voltage-current amplifier
  \item OTA + class AB output = voltage output
  \item voltage gain of OTA depends on load resistance
  \item OTA's and opamps are used with feedback
\end{itemize}

\medskip

\cite{huijsing-opamp-17} says that a simple class A complete general opamp
is called \gls{ota}.
A voltage-to-current converter is called a \gls{ofa}.

\medskip

\cite[subsection 10.5.4]{wicht-pmic-24} says that an \gls{ota} is easier to implement 
than an opamp, because it does not require a power amplifier output stage.

\medskip

PSRR: \cite{steyaert-psrrota-90}

\printbibliography
\end{document}